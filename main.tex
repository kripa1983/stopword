\documentclass[sigconf]{acmart}

\usepackage{booktabs} % For formal tables


\usepackage{booktabs} % For formal tables
\usepackage{caption}
\usepackage{subfig}
\usepackage{url}
\usepackage{epstopdf}
\setcopyright{rightsretained}
%\usepackage{graphics,graphicx}
%\usepackage{cite}%
%\usepackage{epsfig}
\usepackage{array}
\usepackage{pifont}
\usepackage{multirow}
%\usepackage{fixltx2e}
\usepackage{amssymb}
\usepackage{fontspec}
\usepackage{times}


% DOI
\acmDOI{}

% ISBN
\acmISBN{}

%Conference
\acmConference[CIKM'17]{ACM CIKM conference}{November 2017}{Pan Pacific, Singapore} 
\acmYear{2017}
\copyrightyear{2017}

\acmPrice{15.00}


\begin{document}
\title{Stopword Removal: Why Bother? A Case Study on Verbose Queries}
%\titlenote{Produces the permission block, and
%  copyright information}
%\subtitle{Extended Abstract}
%\subtitlenote{The full version of the author's guide is available as
%  \texttt{acmart.pdf} document}


%\author{Ben Trovato}
%\authornote{Dr.~Trovato insisted his name be first.}
%\orcid{1234-5678-9012}
%\affiliation{%
%  \institution{Institute for Clarity in Documentation}
%  \streetaddress{P.O. Box 1212}
%  \city{Dublin} 
%  \state{Ohio} 
%  \postcode{43017-6221}
%}
%\email{trovato@corporation.com}



\begin{abstract}
Stopword removal has traditionally been an integral step in information retrieval preprocessing. In this paper we question the utility of this step for verbose queries on standard datasets. We show in seven FIRE test collections in four languages -- Bangla, Hindi, Gujarati and Marathi, that stopword removal does not lead to noticeable difference in retrieval performance as opposed to not removing stopwords. However, for European languages like English (TREC678 Ad Hoc) and French (CLEF 2005 to 2007), stopword removal leads to a statistically significant drop in performance.
\end{abstract}

%
% The code below should be generated by the tool at
% http://dl.acm.org/ccs.cfm
% Please copy and paste the code instead of the example below. 
%
\begin{CCSXML}
<ccs2012>
<concept>
<concept_id>10002951.10003317</concept_id>
<concept_desc>Information systems~Information retrieval</concept_desc>
<concept_significance>500</concept_significance>
</concept>
</ccs2012>
\end{CCSXML}

\ccsdesc[500]{Information systems~Information retrieval}


\keywords{Stopword removal, retrieval, test collections}


\maketitle
%\input{introduction}
%\input{related}
%\input{methodology}
%\input{results}
%\input{conclusion}

\bibliographystyle{ACM-Reference-Format}
\bibliography{kripa_references} 

\end{document}
