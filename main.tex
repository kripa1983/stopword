\documentclass[sigconf]{acmart}

\usepackage{booktabs} % For formal tables
\usepackage{caption}
\usepackage{subfig}
\usepackage{url}
\usepackage{epstopdf}
\setcopyright{rightsretained}
%\usepackage{graphics,graphicx}
%\usepackage{cite}%
%\usepackage{epsfig}
\usepackage{array}
\usepackage{pifont}
\usepackage{multirow}
%\usepackage{fixltx2e}
\usepackage{amssymb}
\usepackage{fontspec}


% DOI
\acmDOI{}

% ISBN
\acmISBN{}

%Conference
\acmConference[CIKM'17]{ACM CIKM conference}{November 2017}{Pan Pacific, Singapore} 
\acmYear{2017}
\copyrightyear{2017}

\acmPrice{15.00}


\begin{document}
\title{Stopword Removal: Why Bother? A Case Study on Verbose Queries}
%\titlenote{Produces the permission block, and
%  copyright information}
%\subtitle{Extended Abstract}
%\subtitlenote{The full version of the author's guide is available as
%  \texttt{acmart.pdf} document}


%\author{Ben Trovato}
%\authornote{Dr.~Trovato insisted his name be first.}
%\orcid{1234-5678-9012}
%\affiliation{%
%  \institution{Institute for Clarity in Documentation}
%  \streetaddress{P.O. Box 1212}
%  \city{Dublin} 
%  \state{Ohio} 
%  \postcode{43017-6221}
%}
%\email{trovato@corporation.com}



\begin{abstract}
Stopword removal has traditionally been an integral step in information retrieval preprocessing. In this paper we question the utility of this step for verbose queries on standard datasets. We show in seven FIRE test collections in four languages -- Bangla, Hindi, Gujarati and Marathi, that stopword removal does not lead to noticeable difference in retrieval performance as opposed to not removing stopwords. However, for European languages like English (TREC678 Ad Hoc) and French (CLEF 2005 to 2007), stopword removal leads to a statistically significant drop in performance.
\end{abstract}

%
% The code below should be generated by the tool at
% http://dl.acm.org/ccs.cfm
% Please copy and paste the code instead of the example below. 
%
\begin{CCSXML}
<ccs2012>
<concept>
<concept_id>10002951.10003317</concept_id>
<concept_desc>Information systems~Information retrieval</concept_desc>
<concept_significance>500</concept_significance>
</concept>
</ccs2012>
\end{CCSXML}

\ccsdesc[500]{Information systems~Information retrieval}


\keywords{Stopword removal, retrieval, test collections}


\maketitle
%\input{introduction}
%\input{related}
%\input{methodology}
%\input{results}
%\input{conclusion}


\begin{table*}
 \begin{tabular}{|c|c|c|}
 \hline
 \textbf{Dataset}  & \textbf{Number of documents} & \textbf{Number of topics}\\
 \hline
 \hline
FIRE 2010 Ad Hoc Bangla & 123047 & 50\\ \hline
FIRE 2011 Ad Hoc Bangla & 377104 & 50\\ \hline
FIRE 2012 Ad Hoc Bangla & 377111 & 50\\ \hline
\hline

FIRE 2010 Ad Hoc Hindi & 149482 & 50\\ \hline
FIRE 2011 Ad Hoc Hindi & 331599 & 50\\ \hline
\hline
FIRE 2011 Ad Hoc Gujarati & 313163 & 50\\ \hline
\hline
FIRE 2010 Ad Hoc Marathi & 99275 & 50\\ \hline
\hline
TREC678 Ad Hoc English & 528155 & 150\\ \hline
\hline
CLEF 2005 to 2007 Ad Hoc French & 177452 & 148\\
\hline
  
\end{tabular}

\caption{Datasets.}
\end{table*}

\begin{table*}
 \begin{tabular}{|c|c|}
 \hline
 \textbf{Language}  & \textbf{Source}\\
 \hline
 \hline
Bangla & \url{http://www.isical.ac.in/~fire/data/stopwords_list_ben.txt}\\
\hline
Hindi & \url{http://www.isical.ac.in/~fire/data/stopwords_list_hin.txt}\\
\hline
Gujarati & \url{http://irlab.daiict.ac.in/downloads/gujarati_stop_words.zip}\\ 
\hline
Marathi & \url{http://members.unine.ch/jacques.savoy/clef/marathiST.txt}\\
\hline
English & \url{http://www.lemurproject.org/stopwords/stoplist.dft}\\
\hline
French & \url{http://members.unine.ch/jacques.savoy/clef/frenchST.txt}\\
\hline
  
\end{tabular}

\caption{Stopword sources.}
\end{table*}


\begin{table*}
 \begin{tabular}{|c|c|c|c|}
 \hline
 \textbf{Dataset}  & \textbf{Language} & \textbf{With stopwords} & 
\textbf{Stopwords removed}\\
 \hline
 \hline
FIRE 2010 & Bangla & \textbf{0.4337} & 0.4334\\ \hline
FIRE 2011 & Bangla & \textbf{0.2910} & 0.2898\\ \hline
FIRE 2012 & Bangla & 0.2465 & \textbf{0.2472}\\ \hline
\hline

FIRE 2010 & Hindi & \textbf{0.4695} & 0.4569\\ \hline
FIRE 2011 & Hindi & 0.1639 & \textbf{0.1642}\\ \hline
\hline
FIRE 2011 & Gujarati & \textbf{0.2818} & 0.2753\\ \hline
\hline
FIRE 2010 & Marathi & \textbf{0.2587} & 0.2552\\ \hline
\hline
TREC678 & English & $\textbf{0.2150}^*$ & 0.1971\\ \hline
\hline
CLEF 2005 to 2007 & French & $\textbf{0.2739}^*$ & 0.2581\\
\hline
  
\end{tabular}

\caption{This table reports the retrieval performance in terms of Mean Average Precision. For each dataset the better performance value is shown in bold font. * indicates that the performance difference is statistically significant ($p$ $<$ 0.05) at 95\% confidence interval.}
\label{tab:results}
\end{table*}

\bibliographystyle{ACM-Reference-Format}
\bibliography{kripa_references} 

\end{document}
